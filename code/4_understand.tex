\documentclass[10pt,]{tufte-handout}

% ams
\usepackage{amssymb,amsmath}

\usepackage{ifxetex,ifluatex}
\usepackage{fixltx2e} % provides \textsubscript
\ifnum 0\ifxetex 1\fi\ifluatex 1\fi=0 % if pdftex
  \usepackage[T1]{fontenc}
  \usepackage[utf8]{inputenc}
\else % if luatex or xelatex
  \makeatletter
  \@ifpackageloaded{fontspec}{}{\usepackage{fontspec}}
  \makeatother
  \defaultfontfeatures{Ligatures=TeX,Scale=MatchLowercase}
  \makeatletter
  \@ifpackageloaded{soul}{
     \renewcommand\allcapsspacing[1]{{\addfontfeature{LetterSpace=15}#1}}
     \renewcommand\smallcapsspacing[1]{{\addfontfeature{LetterSpace=10}#1}}
   }{}
  \makeatother

\fi

% graphix
\usepackage{graphicx}
\setkeys{Gin}{width=\linewidth,totalheight=\textheight,keepaspectratio}

% booktabs
\usepackage{booktabs}

% url
\usepackage{url}

% hyperref
\usepackage{hyperref}

% units.
\usepackage{units}


\setcounter{secnumdepth}{-1}

% citations

% pandoc syntax highlighting

% longtable

% multiplecol
\usepackage{multicol}

% strikeout
\usepackage[normalem]{ulem}

% morefloats
\usepackage{morefloats}


% tightlist macro required by pandoc >= 1.14
\providecommand{\tightlist}{%
  \setlength{\itemsep}{0pt}\setlength{\parskip}{0pt}}

% title / author / date
\title{Heart Rate Variability and Psychosocial Stress}
\author{Anish Shah, MD}
\date{}

\usepackage{dcolumn}
\usepackage{float}

\begin{document}

\maketitle




\hypertarget{overview}{%
\section{OVERVIEW}\label{overview}}

\hypertarget{disclosures-and-funding}{%
\subsection{Disclosures and Funding}\label{disclosures-and-funding}}

\begin{itemize}
\tightlist
\item
  Emory University
\item
  Atherosclerosis Risk in Communities Study
\item
  No conflicts of interest
\end{itemize}

\hypertarget{atherosclerosis-risk-in-communities-study}{%
\subsection{Atherosclerosis Risk in Communities
Study}\label{atherosclerosis-risk-in-communities-study}}

The ARIC study is a prospective epidemiologic study designed to
investigate atherosclerosis and its clinical outcomes by cardiovascular
risk factors, diseases, and demographics. It involved longitudinal
follow-up over five visits from 1987 to 2013, with outcomes updated as
of 2015, with over 15k initial participants. It collected measures of
psychosocial stress (increased anger, increased vital exhaustion,
decreased social support) and HRV over time.

\hypertarget{background}{%
\subsection{Background}\label{background}}

Psychosocial stressors, such as fatigue and vital exhaustion, are
repeatedly shown to be associated independently with
MACE.\textsuperscript{\protect\hyperlink{ref-Appels1988}{1},\protect\hyperlink{ref-Bogle2018}{2}}
Stressors and depression are shown to correlate with lower
HRV.\textsuperscript{\protect\hyperlink{ref-Huang2018}{3}--\protect\hyperlink{ref-Shah2013}{5}}
The relationship between autonomic dysfunction and psychosocial
stressors is still being studied.\footnote{What is the interaction
  between HRV and psychosocial stressors?}

\hypertarget{purpose}{%
\subsection{Purpose}\label{purpose}}

Objectives: - Examine HRV at V1 and V4 - Study cross-sectional
relationship of stress with outcomes - Compare longitudinal pattern of
anger from V2 and V4 - Study effect of stress on HRV - Adjust for gender
effect on HRV

Hypothesis:

\begin{itemize}
\tightlist
\item
  Increased stress will associate with lower HRV
\item
  Changes in anger will associate with proportional changes in HRV
\item
  Gender will significantly interact with the stress and HRV
\end{itemize}

\hypertarget{methods}{%
\subsection{Methods}\label{methods}}

\begin{itemize}
\tightlist
\item
  Description tables
\end{itemize}

\hypertarget{hrv-description}{%
\section{HRV description}\label{hrv-description}}

\hypertarget{vist-1-hrv-measures}{%
\subsection{Vist 1 HRV measures}\label{vist-1-hrv-measures}}

\includegraphics{4_understand_files/figure-latex/unnamed-chunk-3-1}

\hypertarget{visit-4-hrv-measures}{%
\subsection{Visit 4 HRV measures}\label{visit-4-hrv-measures}}

\includegraphics{4_understand_files/figure-latex/unnamed-chunk-5-1}

\hypertarget{stress-description}{%
\section{Stress description}\label{stress-description}}

\hypertarget{visit-2-stress-and-measures}{%
\subsection{Visit 2 Stress and
measures}\label{visit-2-stress-and-measures}}

\includegraphics{4_understand_files/figure-latex/unnamed-chunk-7-1}

\hypertarget{visit-4-stress-measures}{%
\subsection{Visit 4 Stress measures}\label{visit-4-stress-measures}}

\% latex table generated in R 3.5.0 by xtable 1.8-2 package \% Sun Oct
21 21:34:24 2018

\begin{table}[ht]
\centering
\caption{Visit 4 stress measures} 
\begingroup\scriptsize
\begin{tabular}{llllllllll}
  \hline
  \hline
 &  & GENDER &  & HTN &  & CHD &  & DIABETES &  \\ 
   &  & Female & Male & 0 & 1 & 0 & 1 & 0 & 1 \\ 
   & n & 6389 & 5048 & 6030 & 5407 & 10305 & 1132 & 9526 & 1911 \\ 
  SPIELBERGER & mean & 15.1 & 15.4 & 15.2 & 15.2 & 15.2 & 15.5 & 15.2 & 15.4 \\ 
  anger\_level & mean &  1.6 &  1.6 &  1.6 &  1.6 &  1.6 &  1.6 &  1.6 &  1.6 \\ 
   \hline
\end{tabular}
\endgroup
\end{table}

\hypertarget{longitudinal-anger-comparison}{%
\subsubsection{Longitudinal anger
comparison}\label{longitudinal-anger-comparison}}

\includegraphics{4_understand_files/figure-latex/unnamed-chunk-9-1}

\hypertarget{relationships}{%
\section{RELATIONSHIPS}\label{relationships}}

\hypertarget{hrv-from-v1-to-stress-at-v2}{%
\subsection[HRV from V1 to Stress at V2]{\texorpdfstring{HRV from V1 to
Stress at V2\footnote{Does HRV predict stress levels in the future?}}{HRV from V1 to Stress at V2}}\label{hrv-from-v1-to-stress-at-v2}}

\hypertarget{fitting-regression-models}{%
\subsection{Fitting regression models}\label{fitting-regression-models}}

\% Table created by stargazer v.5.2.2 by Marek Hlavac, Harvard
University. E-mail: hlavac at fas.harvard.edu \% Date and time: Sun, Oct
21, 2018 - 21:35:18 \% Requires LaTeX packages: dcolumn

\begin{table}[H] \centering 
  \caption{LM fits for Stress ~ HRV + Covar} 
  \label{} 
\scriptsize 
\begin{tabular}{@{\extracolsep{0pt}}lD{.}{.}{-2} D{.}{.}{-2} D{.}{.}{-2} D{.}{.}{-2} D{.}{.}{-2} D{.}{.}{-2} D{.}{.}{-2} } 
\\[-1.8ex]\hline 
\hline \\[-1.8ex] 
 & \multicolumn{7}{c}{\textit{Dependent variable:}} \\ 
\cline{2-8} 
\\[-1.8ex] & \multicolumn{1}{c}{ISEL} & \multicolumn{1}{c}{AP} & \multicolumn{1}{c}{TA} & \multicolumn{1}{c}{BE} & \multicolumn{1}{c}{LSNS} & \multicolumn{1}{c}{MAASTRICHT} & \multicolumn{1}{c}{SPIELBERGER} \\ 
\hline \\[-1.8ex] 
 BMI &  & 0.01^{***} &  & -0.01^{**} & 0.04^{***} & 0.07^{***} & 0.05^{***} \\ 
  AGE\_V1 &  &  & -0.004^{**} &  &  & 0.01 & -0.05^{***} \\ 
  RACEBlack &  & 0.50 & 0.17 & -0.20 & 0.99 & 0.93 & -1.56^{**} \\ 
  RACEWhite &  & 0.56^{*} & 0.37 & -0.52 & 2.21^{*} & 0.58 & -1.02 \\ 
  DRINKERFormer &  & 0.14^{***} &  &  &  & 0.60^{***} & -0.20^{*} \\ 
  DRINKERNever &  & 0.09^{**} &  &  &  & 0.72^{***} & -0.65^{***} \\ 
  SMOKERFormer & 0.23^{***} & -0.13^{***} & 0.07^{**} & 0.29^{***} & 0.90^{***} & -1.10^{***} & -0.42^{***} \\ 
  SMOKERNever & 0.20^{***} & -0.18^{***} & 0.17^{***} & 0.24^{***} & 1.15^{***} & -0.91^{***} & -1.06^{***} \\ 
  HTN1 &  & 0.14^{***} & -0.07^{**} & -0.11^{***} & -0.22^{*} & 0.23^{***} &  \\ 
  CHD1 &  &  & -0.13^{**} &  &  & 0.52^{***} & 0.53^{***} \\ 
  DIABETES1 & -0.14 & 0.08 & -0.11^{***} & -0.15^{**} & -0.77^{***} & 0.77^{***} &  \\ 
  HFc & 0.004 & 0.003^{**} &  &  &  & 0.003^{***} & 0.01^{*} \\ 
  LFc & 0.004^{**} & 0.002^{**} &  &  & 0.004^{*} &  & 0.004^{*} \\ 
  TPc & -0.004^{*} & -0.002^{**} & -0.0002 &  & -0.002^{**} &  & -0.004 \\ 
  VLFc &  &  &  &  &  & 0.01^{*} &  \\ 
  LF\_HF &  &  & -0.01^{***} & 0.01^{*} &  & -0.05^{***} & 0.02 \\ 
  SDNNc &  &  &  &  &  & -0.02^{***} & -0.004 \\ 
  PNN50 &  &  &  &  &  & 0.02^{***} & 0.01 \\ 
 \hline \\[-1.8ex] 
Observations & \multicolumn{1}{c}{11,396} & \multicolumn{1}{c}{11,396} & \multicolumn{1}{c}{11,396} & \multicolumn{1}{c}{11,396} & \multicolumn{1}{c}{11,396} & \multicolumn{1}{c}{11,396} & \multicolumn{1}{c}{11,396} \\ 
Log Likelihood & \multicolumn{1}{c}{-29,670.00} & \multicolumn{1}{c}{-21,303.00} & \multicolumn{1}{c}{-18,742.00} & \multicolumn{1}{c}{-23,882.00} & \multicolumn{1}{c}{-36,843.00} & \multicolumn{1}{c}{-32,133.00} & \multicolumn{1}{c}{-31,610.00} \\ 
Akaike Inf. Crit. & \multicolumn{1}{c}{59,356.00} & \multicolumn{1}{c}{42,638.00} & \multicolumn{1}{c}{37,510.00} & \multicolumn{1}{c}{47,786.00} & \multicolumn{1}{c}{73,709.00} & \multicolumn{1}{c}{64,305.00} & \multicolumn{1}{c}{63,258.00} \\ 
\hline 
\hline \\[-1.8ex] 
\textit{Note:}  & \multicolumn{7}{r}{$^{*}$p$<$0.1; $^{**}$p$<$0.05; $^{***}$p$<$0.01} \\ 
\end{tabular} 
\end{table}

\hypertarget{conclusion}{%
\section{CONCLUSION}\label{conclusion}}

\hypertarget{references}{%
\subsection*{References}\label{references}}
\addcontentsline{toc}{subsection}{References}

\hypertarget{refs}{}
\leavevmode\hypertarget{ref-Appels1988}{}%
1. Appels A, Mulder P. Excess fatigue as a precursor of myocardial
infarction. \emph{European Heart Journal}. 1988;9(7):758-764.
doi:\href{https://doi.org/10.1093/eurheartj/9.7.758}{10.1093/eurheartj/9.7.758}

\leavevmode\hypertarget{ref-Bogle2018}{}%
2. Bogle BM, Sotoodehnia N, Kucharska-Newton AM, Rosamond WD. Vital
exhaustion and sudden cardiac death in the Atherosclerosis Risk in
Communities Study. \emph{Heart (British Cardiac Society)}.
2018;104(5):423-429.
doi:\href{https://doi.org/10.1136/heartjnl-2017-311825}{10.1136/heartjnl-2017-311825}

\leavevmode\hypertarget{ref-Huang2018}{}%
3. Huang M, Shah A, Su S, et al. Association of Depressive Symptoms and
Heart Rate Variability in Vietnam War--Era Twins. \emph{JAMA
Psychiatry}. 2018;75(7):705.
doi:\href{https://doi.org/10.1001/jamapsychiatry.2018.0747}{10.1001/jamapsychiatry.2018.0747}

\leavevmode\hypertarget{ref-Vroege2012}{}%
4. Vroege EM, Zuidersma M, De Jonge P. Vital exhaustion and somatic
depression: The same underlying construct in patients with myocardial
infarction? \emph{Psychosomatic Medicine}. 2012;74(5):446-451.
doi:\href{https://doi.org/10.1097/PSY.0b013e31825a7194}{10.1097/PSY.0b013e31825a7194}

\leavevmode\hypertarget{ref-Shah2013}{}%
5. Shah AJ, Lampert R, Goldberg J, Veledar E, Bremner JD, Vaccarino V.
Posttraumatic stress disorder and impaired autonomic modulation in male
twins. \emph{Biological Psychiatry}. 2013;73(11):1103-1110.
doi:\href{https://doi.org/10.1016/j.biopsych.2013.01.019}{10.1016/j.biopsych.2013.01.019}



\end{document}
